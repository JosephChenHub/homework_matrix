\documentclass[12pt,a4paper]{article}
\usepackage[utf8]{inputenc}
\usepackage{amsmath}
\usepackage{amsfonts}
\usepackage{amssymb}
\usepackage{graphicx}
\usepackage{geometry}
\usepackage{cases}
\usepackage{fontspec}
\setmainfont{Times New Roman}
\usepackage{enumerate}
\usepackage{bm}

\geometry{left=2.5cm,right=2.5cm,top=2cm,bottom=2cm}
\makeatletter
\renewcommand{\thesection}{\arabic{section}.} 
%
\newcommand{\mysection}[2]{
\section{Problem #1 of Page #2}	
	}

%\renewcommand{\thesubsection}{(\arabic{subsection})} 

\makeatother


\title{homework1}
\author{Zuyao Chen 201728008629002 \\ zychen.uestc@gmail.com}

\date{}


\begin{document}
\maketitle
\mysection{3.(2)}{25}
yes. 
Call the set of all real symmetric matrix ``${S}$''.
\begin{itemize}
	\item obviously zero matrix $\bm O \in {S}$;
	\item $\forall \bm A \in {S},\forall \alpha \in \mathbf{R}$, 
			$(\alpha \bm{A})^{T} = \alpha \bm{A}$, $\alpha \bm{A}$ is still symmetric,
			so $\alpha \bm{A} \in  {S}$;
	\item $\forall \bm{A} , \bm{B} \in  {S}$, $(\bm{A}+\bm{B})^{T} = (\bm{A}+\bm{B})$,
			$(\bm{A}+\bm{B})$ is symmetric, $(\bm{A}+\bm{B}) \in  {S}$		
\end{itemize}
in summary,	the set of all real symmetric matrix is closed under addition and scalar multiplication, so it's a linear space.
\mysection{4}{25}
\textbf{Proof:} \\
\indent Let
$a_1 \cdot 1 + a_2 \cdot \cos^{2}t + a_3 \cdot \cos 2t = 0,a_1,a_2,a_3 \in \mathbf{R}$.
Suppose that $1, \cos^{2} t, \cos 2t$ is linear independent,then 
it must has $ a_1 = a_2 = a_3 = 0$.
But as we all know,$\cos 2t = 2\cos^{2} t - 1$,
if $a_1 = 1, a_2 = -2 , a_3 = 1$, it also has $a_1 \cdot 1 + a_2 \cdot \cos^{2}t + a_3 \cdot \cos 2t = 0$,which is contrary to the hypothesis.
So $1, \cos^{2} t, \cos 2t$ is linear dependent.

\mysection{6}{25}
let $(\eta_1,\eta_2,\eta_3)^{T}$ be the new coordinates of vector $\bm{x}$,
then 
$\bm{x}= \eta_1\bm{x}_1 + \eta_2\bm{x}_2+ \eta_3\bm{x}_3 $,
which equals to 
\[
(\bm{x}_1^{T},\bm{x}_2^{T},\bm{x}_3^{T}) (\eta_1,\eta_2,\eta_3)^{T} = \bm{x}^{T}
\]
so $(\eta_1,\eta_2,\eta_3)^{T} 
= (\bm{x}_1^{T},\bm{x}_2^{T},\bm{x}_3^{T})^{-1}\bm{x}^{T} 
= (33,-82,154)^{T}$.\\
new coordinates of $\bm{x}$ : $(33,-82,154)^{T}$

\mysection{8}{25}
\begin{enumerate}[(1)] 
	\item 

the original equation is equivalent to 
\begin{numcases}{}
(\bm{x}_1,\bm{x}_2,\bm{x}_3,\bm{x}_4) (1,2,0,0)^{T} = (\bm{y}_1,\bm{y}_2,\bm{y}_3,\bm{y}_4)  (0,0,1,0)^{T} \label{form1} \\
(\bm{x}_1,\bm{x}_2,\bm{x}_3,\bm{x}_4)  (0,1,2,0)^{T} = (\bm{y}_1,\bm{y}_2,\bm{y}_3,\bm{y}_4)  (0,0,0,1)^{T}  \label{form2}\\
(\bm{y}_1,\bm{y}_2,\bm{y}_3,\bm{y}_4)  (1,2,0,0)^{T} = (\bm{x}_1,\bm{x}_2,\bm{x}_3,\bm{x}_4)  (0,0,1,0)^{T} \label{form3}\\
(\bm{y}_1,\bm{y}_2,\bm{y}_3,\bm{y}_4)  (0,1,2,0)^{T} = (\bm{x}_1,\bm{x}_2,\bm{x}_3,\bm{x}_4)  (0,0,0,1)^{T} \label{form4}
\end{numcases}
combining equation(\ref{form1}),(\ref{form2}),(\ref{form3}) and (\ref{form4}),we have \\
\begin{numcases}{ }
(\bm{y}_1,\bm{y}_2,\bm{y}_3,\bm{y}_4) (1,0,0,0)^{T} = (\bm{x}_1,\bm{x}_2,\bm{x}_3,\bm{x}_4) (4,8,1,-2)^{T} \\
(\bm{y}_1,\bm{y}_2,\bm{y}_3,\bm{y}_4) (0,1,0,0)^{T} =(\bm{x}_1,\bm{x}_2,\bm{x}_3,\bm{x}_4) (-2,-4,0,1)^{T} \\
(\bm{y}_1,\bm{y}_2,\bm{y}_3,\bm{y}_4) (0,0,1,0)^{T} = (\bm{x}_1,\bm{x}_2,\bm{x}_3,\bm{x}_4) (1,2,0,0)^{T} \\
(\bm{y}_1,\bm{y}_2,\bm{y}_3,\bm{y}_4) (0,0,0,1)^{T} = (\bm{x}_1,\bm{x}_2,\bm{x}_3,\bm{x}_4) (0,1,2,0)^{T} 
\end{numcases}
thus the transformation matrix ($(\bm{y}_1,\bm{y}_2,\bm{y}_3,\bm{y}_4) =(\bm{x}_1,\bm{x}_2,\bm{x}_3,\bm{x}_4) \bm{C}$):\\
\[
\bm{C} = \left(
\begin{array}{cccc}
4 &-2 &1 &0\\
8 &-4 &2 &1\\
1 &0 & 0 &2 \\
-2&1 & 0 &0 
\end{array}
\right)
\]	
\item let $\bm{z}$ be the new coordinates, \\
\[  
(\bm{y}_1,\bm{y}_2,\bm{y}_3,\bm{y}_4)(2,-1,1,1)^{T}
=(\bm{x}_1,\bm{x}_2,\bm{x}_3,\bm{x}_4)\bm{C}(2,-1,1,1)^{T}
\]
thus,$\bm{z} = \bm{C}(2,-1,1,1)^{T} = (11,23,4,-5)^{T} $.
\end{enumerate}
\mysection{10}{26}
call the span space ``${S}$'',the linear combination of 
$\bm{y}_1,\bm{y}_2,\bm{y}_3$ can be written as
\begin{equation} 
\begin{split}
&k_1 (\bm{x}_1 - 2\bm{x}_2 + 3\bm{x}_3)+
 k_2(2\bm{x}_1 + 3\bm{x}_2 + 2\bm{x}_3)
 + k_3(4\bm{x}_1 + 13\bm{x}_2) \\
 =& (k_1 + 2k_2 + 4k_3)\bm{x}_1 + 
 (-2k_1+3k_2 + 13k_3)\bm{x}_2 +
 (3k_1 + 2k_2)\bm{x}_3 \\
\end{split}
\end{equation}
it's a linear combination of $\bm{x}_1,\bm{x}_2,\bm{x}_3$,
thus $\bm{x}_1,\bm{x}_2,\bm{x}_3$ is one basis of space ${S}$.
\mysection{11}{26} 
${S} = {V}_1 \bigcap  {V}_2 =
\{(\xi_1,\xi_2,\xi_3,\xi_4)|\xi_1 = -\xi_3, \xi_2 = -\xi_4
\}
$,let $\bm{e}_1,\bm{e}_2,\bm{e}_3,\bm{e}_4$
be the standard basis of $\mathbf{R}^{4}$.
$\forall \bm{x} = (x_1,x_2,x_3,x_4) \in {S}$, \\
\begin{equation}
\begin{split}
\bm{x} &= x_1 \bm{e}_1 + x_2\bm{e}_2 + x_3\bm{e}_3 + x_4\bm{e}_4 \\
&= x_1(\bm{e}_1 -\bm{e}_3) + x_2(\bm{e}_2 - \bm{e}_4) \\
& = x_1(1,0,-1,0)^{T} + x_2 (0,1,0,-1)^{T}
\end{split} 
\end{equation} 
any element in ${S}$ can be derived from linear combination of
$\bm{e}_1 - \bm{e}_3$ and $\bm{e}_2 - \bm{e}_4$,
so $\bm{e}_1 - \bm{e}_3$, $\bm{e}_2 - \bm{e}_4$ is one basis of ${S}$.
\mysection{12}{26}
\begin{enumerate}[(1)]
\item \textbf{Proof}: 
\begin{itemize}
	\item obviously zero matrix $\bm O \in {V}$ 
	\item $\forall \bm{A},\bm{B} \in {V},\forall \alpha,\beta \in \mathbf{R},
		\alpha \bm{A} + \beta \bm{B} = \left(
		\begin{array}{cc}
		 \alpha a_{11} + \beta b_{11} & * \\
		 * & \alpha a_{22} + \beta b_{22}
		\end{array} 
		\right) \in {V}
	$
\end{itemize}
${V}$ is closed under addition and scalar multiplication ,
thus it is a subspace of $\mathbf{R}^{2\text{X}2}$.
\item let $ 
\bm{e}_1 = \left(
	\begin{array}{cc}
	 1 & 0  \\
	 0 & -1  
	\end{array} 
\right),
\bm{e}_2 = \left(
\begin{array}{cc}
0 & 1  \\
0 & 0  
\end{array} 
\right),
\bm{e}_3 = \left(
\begin{array}{cc}
0 & 0  \\
1 & 0  
\end{array} 
\right)
$,
$\forall A \in {V}$, \\
\[
A = \left( \begin{array}{cc}
a_{11} & a_{12} \\
a_{21} & -a_{11}
\end{array} \right) 
= a_{11}\bm{e}_1 + a_{12}\bm{e}_2 + 
a_{21}\bm{e}_3
\]
$\bm{e}_1,\bm{e}_2,\bm{e}_3$ is linear independent,
so the dimension of subspace ${V}$ is 3
and $\bm{e}_1,\bm{e}_2,\bm{e}_3$ is one basis of ${V}$.

\end{enumerate}
\section{Additional Problem}
\textbf{Question}: \\
if ${W}_1,{W}_2,{W}_3$ are subspaces of ${W}$, then 
$({W}_1 \bigcap {W}_3) + ({W}_2 \bigcap {W}_3) \subset ({W}_1 +{W}_2) \bigcap {W}_3$.
Can the left equal to the right under some circumstances?
 


\mysection{1.(2)}{77}
Yes. $\forall \bm{X}_1 ,\bm{X}_2 \in \mathbf{R}^{n\text{x}n}$,$\forall 
\alpha,\beta \in \mathbf{R}
$\\
\[T(\alpha \bm{X}_1 + \beta \bm{X}_2)
 = \bm{B}(\alpha \bm{X}_1 + \beta \bm{X}_2)\bm{C} 
 = \alpha \bm{B}\bm{X}_1\bm{C} + \beta \bm{BX}_2\bm{C} = \alpha T(\bm{X}_1) + \beta T(\bm{X}_2) \]
thus $T$ is linear transformation.
\mysection{6}{78}
let $\bm{g} = (\bm{x}_1,\bm{x}_2,..,\bm{x}_6)$,
\begin{equation*}
\left\lbrace 
\begin{split}
\frac{\partial \bm{x}_1}{\partial t} &= a\bm{x}_1-b\bm{x}_2 \\
\frac{\partial \bm{x}_2}{\partial t} &= b\bm{x}_1+a\bm{x}_2\\
\frac{\partial \bm{x}_3}{\partial t} &=\bm{x}_1 + a\bm{x}_3 - b\bm{x}_4 \\
\frac{\partial \bm{x}_4}{\partial t}&=\bm{x}_2 + b\bm{x}_3 + a\bm{x}_4\\
\frac{\partial \bm{x}_5}{\partial t}&=\bm{x}_3 + a\bm{x}_5 - b\bm{x}_6\\
\frac{\partial \bm{x}_6}{\partial t}&= \bm{x}_4 +b\bm{x}_5 + a\bm{x}_6 
\end{split}
\right .
\end{equation*}
thus 
\begin{equation*}
\begin{split}
\nabla_t(\bm{g}) &= (\frac{\partial \bm{x}_1}{\partial t},
\frac{\partial \bm{x}_2}{\partial t},...,\frac{\partial \bm{x}_6}{\partial t}) \\
&=\bm{g} \left(
\begin{array}{cccccc}
a & b & 1 &0 &0 &0 \\
-b &a & 0 &1&0 &0 \\
0 &0 &a &b &1 &0\\
0 &0 &-b&a&0 &1\\
0 &0 &0 &0&a&b\\
0 &0 &0 &0 &-b &a
\end{array}
\right)\\
&= \bm{gD}
\end{split}
\end{equation*}
$\bm D$ is what we wanted.
\mysection{8}{78}
\[
\begin{split}
T_1\bm{E}_{11} &=\left(  \begin{array}{cc}
a & 0 \\
c & 0
\end{array}
\right)=a\bm{E}_{11} + c\bm{E}_{21},
T_1\bm{E}_{12} =\left(  \begin{array}{cc}
0 & a \\
0 & c
\end{array}
\right)
= a\bm{E}_{12} + c\bm{E}_{22},  \\
T_1\bm{E}_{21} &=\left(  \begin{array}{cc}
b & 0 \\
d & 0
\end{array}
\right)
=b\bm{E}_{11} + d\bm{E}_{21}, 
T_1\bm{E}_{22} =\left(  \begin{array}{cc}
0 & b \\
0 & d
\end{array}
\right)
=b\bm{E}_{12} + d\bm{E}_{22}
\end{split}
\]
\begin{equation*}
\begin{split}
T_1(\bm{E}_{11},\bm{E}_{12},\bm{E}_{21},\bm{E}_{22}) 
&= (T_1\bm{E}_{11},T_1\bm{E}_{12},T_1\bm{E}_{21},T_1\bm{E}_{22}) \\
&= (\bm{E}_{11},\bm{E}_{12},\bm{E}_{21},\bm{E}_{22})
\left( \begin{array}{cccc}
a & 0 & b & 0\\
0 & a & 0 & b\\
c & 0 & d & 0 \\
0 & c & 0 & d
\end{array} 
\right)  \\
\bm{A}_1 &= \left( \begin{array}{cccc}
a & 0 & b & 0\\
0 & a & 0 & b\\
c & 0 & d & 0 \\
0 & c & 0 & d
\end{array} 
\right) 
\end{split}
\end{equation*}
\qquad similarly,we can get 
\begin{equation*}
\bm{A}_2 = \left(\begin{array}{cccc}
a & c & 0 & 0 \\
b & d & 0 & 0 \\
0 & 0 & a & c\\
0 & 0 & b & d
\end{array} 
\right),\bm{A}_3 = \bm{A}_1\bm{A}_2 
\end{equation*}

\mysection{9}{78}
\textbf{Proof:} \\
\indent considering the linear combination of $\bm{x},T\bm{x},...,T^{k-1}\bm{x}$
\[
\alpha_1 \bm{x} + \alpha_2 T\bm{x} + ...+ \alpha_kT^{k-1}\bm{x} = \bm{0},
\]
apply transformation $T$ for both sides,we have
\[
\alpha_1T\bm{x} + \alpha_2T^{2}\bm{x} + ... + \alpha_{k-1}T^{k-1}\bm{x} + \alpha_kT^{k}\bm{x} = \bm{0}
\]
as $T^{k}\bm{x} = \bm{0},T^{k-1}\bm{x} \neq \bm{0}$,  
\[
\alpha_1 T\bm{x} +\alpha_2T^{2}\bm{x} + ... + \alpha_{k-1}T^{k-1}\bm{x} = \bm{0}
\]
repeat the transformation above until 
\[
\alpha_1 T^{k-1}\bm{x} = \bm{0}
\]
so we get $\alpha_1 = 0$,then go back 
\[
\alpha_1 T^{k-2}\bm{x} + \alpha_2 T^{k-1}\bm{x} = \bm{0}
\]
we get $\alpha_2 = 0$,then from 
\[
\alpha_1 T^{k-3}\bm{x} + \alpha_2 T^{k-2}\bm{x} + \alpha_3 T^{k-1}\bm{x} = \bm{0}
\]
we get $\alpha_3 = 0$,
similarly we can get $\alpha_1 = \alpha_2 = ... = \alpha_k = 0$. \\
Thus $\bm{x},T\bm{x},...,T^{k-1}\bm{x}$ is linear independent. 
\mysection{10}{78} 
\[
T\bm{x} = \bm{x}\left(\begin{array}{ccc}
0 & 1 & 0\\
0 & 0 & 1\\
0 & 0 & 0 
\end{array} 
\right) 
\]
thus,
\[
T^{2}\bm{x} = \bm{x}\left( \begin{array}{ccc}
0 & 0 & 1\\
0 & 0 & 0\\
0 & 0 & 0
\end{array} 
\right) 
\]
\[
\begin{split} 
R(T^{2}) &= \left\lbrace \bm{y}|\bm{y}=T^2\bm{x},\bm{x}\in \mathbf{R}^{3} \right\rbrace
= \left\lbrace (0,0,\xi_1)|\xi_1 \in \mathbf{R}  \right\rbrace  \\
 N(T^{2}) &= \left\lbrace \bm{x}|T^{2}\bm{x} = \bm{0} ,\bm{x} \in \mathbf{R}^{3}\right\rbrace 
= \left\lbrace (0,\xi_2,\xi_3)|\xi_2,\xi_3 \in \mathbf{R} \right\rbrace
\end{split} 
\]
for $R(T^{2})$,its dimension is 1 and $(0,0,1)$ is one basis ;\\
for $N(T^{2})$,its dimension is 2 and $(0,1,0),(0,0,1)$ is one basis.
\mysection{11}{78}
\begin{enumerate}[(1)]
\item  $(\bm{y}_1,\bm{y}_2,\bm{y}_3) = (\bm{x}_1,\bm{x}_2,\bm{x}_3) \bm{C}$,
thus 
\[
\begin{split}
\bm{C} &= \left(\begin{array}{ccc}
1 & 2 & 1\\
0 & 1 & 1 \\
1 & 0 & 1
\end{array} \right)^{-1} \left(\begin{array}{ccc}
1 & 2 &2\\
2 & 2 &-1\\
-1& -1&-1
\end{array} 
\right) \\
&= \left(\begin{array}{ccc} 
-2 & -1.5 & 1.5 \\
1 & 1.5 &1.5 \\
1 &0.5&-2.5
\end{array} 
\right) 
\end{split} 
\]
\item  $T(\bm{x}_1,\bm{x}_2,\bm{x}_3) =(\bm{y}_1,\bm{y}_2,\bm{y}_3)
= (\bm{x}_1,\bm{x}_2,\bm{x}_3) \bm{C}$,
 thus the matrix is $\bm{C}$
\item $T(\bm{y}_1,\bm{y}_2,\bm{y}_3) = (\bm{y}_1,\bm{y}_2,\bm{y}_3) \bm{C}$,
thus the matrix is $\bm{C}$
\end{enumerate} 
\mysection{12}{79}
\begin{equation}
\label{form15.1}
\det (\bm{A} - \lambda \bm{I}) = 0
\end{equation}
we can get $\lambda_1 = -2,\lambda_2 = \lambda_3 = 1$ from equation(\ref{form15.1}).\\
for $\lambda_1 = -2 $,solving the equation $(\bm{A} - \lambda_1 \bm{I})\bm{x} = \bm{0}$,
we get a basic solution:$(0,0,1)^{T} $.all eigenvectors of $T$ are 
\[
k\bm{x}_3, k \in \mathbf{R}, k \neq 0
\]
for $\lambda_2 = \lambda_3 = 1$, we get a basic solution:$(3,-6,20)^{T}$,all eigenvectors of $T$
are 
\[
k(3\bm{x}_1 - 6\bm{x}_2 + 20\bm{x}_3), k \in \mathbf{R}, k \neq 0
\]
\mysection{13}{79}
\qquad from $\det(\bm{A}-\lambda\bm{I}) = 0$  we can get $\lambda_1 = 2,\lambda_2 = \lambda_3 =1$.
for $\lambda_1 = 2$, we get a basic solution $(0,0,1)^{T}$;
for $\lambda_2 = \lambda_3 = 1$, we get a basic solution $(-1,-2,1)^{T}$.
then we obtain a non-singular matrix:
\[
\bm{P}_1 = \left( \begin{array}{ccc}
0 & -1 & 1 \\
0 & -2 & 0\\
1 & 1  & 0
\end{array} 
\right) 
\]
further,
\[
\bm{P}_1^{-1}\bm{AP}_1 = \left( \begin{array}{ccc}
2 &    0 &   -1 \\
0 &    1 &    2 \\
0 &    0 &    1
\end{array}  
\right) 
\]
\mysection{15}{79}
let $\psi(\lambda) = (2\lambda^{4} - 12\lambda^{3} + 19 \lambda^{2} - 29\lambda + 37)$,
the characteristic polynomial of A \\
\[
\varphi(\lambda) = \det(\bm{A} - \lambda\bm{I}) = (\lambda - 1)(\lambda - 5) + 2 
= \lambda^{2} - 6\lambda + 7
\]
so 
\[
\psi(\lambda) = \varphi(\lambda)(2\lambda^{2} + 5) + \lambda + 2
\] 
\[
\begin{split}
\psi(\bm{A}) &= \varphi(\bm{A})(2\bm{A}^{2} + 5\bm{I}) + \bm{A} + 2\bm{I} \\
&= \bm{A} + 2\bm{I}
\end{split} 
\] 
the original problem is equivalent to  solving ${\psi(\bm{A})}^{-1} $ ,
\[
\psi(\bm{A})^{-1} = (\bm{A}+2\bm{I})^{-1} = \frac{1}{23}\left( \begin{array}{cc}
7 & 1 \\
-2 & 3
\end{array} 
\right) 
\]
\mysection{16}{79}
\begin{enumerate}[(1)]
	\item the characteristic polynomial 
	\[
		\det(\bm{A} - \lambda\bm{I}) = (\lambda - 9)(\lambda + 9)^{2}
	\]
	the minimum polynomial 
	\[
	m(\lambda) = (\lambda - 9)(\lambda + 9)  
	\]
	\item the characteristic polynomial 
	\[
	\det(\bm{A} - \lambda\bm{I}) = (\lambda^2- 2a_0\lambda + a_0^2 + a_1^2 + a_2^2 + a_3^2)^2
	\]
	the minimum polynomial 
	\[
	m(\lambda) = \lambda^2- 2a_0\lambda + a_0^2 + a_1^2 + a_2^2 + a_3^2
	\]
\end{enumerate}

\mysection{17}{79}
\text{Proof:} \\
\indent $\det(\bm{A} - \lambda\bm{I}) = \det(\bm{A}^{T} - \lambda \bm{I})$,
thus $A$ and $A^{T}$ have the same characteristic polynomial
and same minimum polynomial.
\mysection{18}{79}
\textbf{Proof:}  \\
\indent $V_{\lambda_0} = \left\lbrace \bm{x} | T_1\bm{x} = \lambda_0\bm{x},\bm{x} \in V^{n} \right\rbrace$,
$\forall \bm{x} \in V_{\lambda_0}$, 
\[
T_1(T_2\bm{x}) = T_2(T_1\bm{x}) = T_2(\lambda_0 \bm{x}) = \lambda_0 (T_2\bm{x}),(T_2\bm{x})\in V^{n}
\]
$T_2\bm{x}$ still belongs to $V_{\lambda_0}$,thus $V_{\lambda_0}$ is the invariant subspace of $T_2$.

\section{Additional Problem}
\begin{enumerate}[(1)]
	\item \text{Proof:} \\
	\indent $N(f(T)) = \left\lbrace \bm{x}|f(T)\bm{x} = \bm{0},\bm{x} \in V^{n}\right \rbrace$, 
			$N(g(T)) = \left\lbrace \bm{x}|g(T)\bm{x} = \bm{0},\bm{x} \in V^{n}\right \rbrace$.\\
	two polynomials $f(\lambda)$ and $g(\lambda)$ are relatively prime,thus $f(\lambda) = 0$ and $g(\lambda) = 0$
	have different roots.$\forall \bm{x} \in N(f(T))\bigcap N(g(T))$,
	\begin{numcases}{}
	f(T)\bm{x} = \bm{0} \label{form211}\\
	g(T)\bm{x} = \bm{0}  \label{form212}
	\end{numcases} 
	equation(\ref{form211}) - equation(\ref{form212}),
	\[
	(f(T) - g(T)) \bm{x} = \bm{0} \label{form213}
	\]	
	the equation(\ref{form213}) only has a solution of $\bm{0}$.	
		
\end{enumerate}


\end{document}
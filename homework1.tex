\documentclass[12pt,a4paper]{article}
\usepackage[utf8]{inputenc}
\usepackage{amsmath}
\usepackage{amsfonts}
\usepackage{amssymb}
\usepackage{graphicx}
\usepackage{geometry}
\usepackage{cases}

\geometry{left=2.5cm,right=2.5cm,top=2.5cm,bottom=2.5cm}

\title{homework1}
\author{Zuyao Chen 201728008629002 \\ zychen.uestc@gmail.com}

\date{}

\begin{document}
\maketitle
\section{Problem 3.(2) of Page 25}
yes. 
Call the set of all real symmetric matrix ``$\mathcal{S}$''.
\begin{itemize}
	\item obviously zero matrix $O \in \mathcal{S}$;
	\item $\forall A \in \mathcal{S},\forall \alpha \in \mathcal{R}$, 
			$(\alpha A)^{T} = \alpha A$, $\alpha A$ is still symmetric,
			so $\alpha A \in \mathcal{S}$;
	\item $\forall A , B \in \mathcal{S}$, $(A+B)^{T} = (A+B)$,
			$(A+B)$ is symmetric, $(A+B) \in \mathcal{S}$		
\end{itemize}
in summary,	the set of all real symmetric matrix is closed under addition and scalar multiplication, so it's a linear space.
\section{Problem 4 of Page 25}
\textbf{Proof:} \\
\indent Let
$a_1 \cdot 1 + a_2 \cdot \cos^{2}t + a_3 \cdot \cos 2t = 0,a_1,a_2,a_3 \in \mathcal{R}$.
Suppose $1, \cos^{2} t, \cos 2t$ is linear independent,then 
it must has $ a_1 = a_2 = a_3 = 0$.
But as we all know,$\cos 2t = 2\cos^{2} t - 1$,
if $a_1 = 1, a_2 = -2 , a_3 = 1$, it also has $a_1 \cdot 1 + a_2 \cdot \cos^{2}t + a_3 \cdot \cos 2t = 0$,which is contrary to the hypothesis.
So $1, \cos^{2} t, \cos 2t$ is linear dependent.

\section{Problem 6 of Page 25}
let $(\eta_1,\eta_2,\eta_3)$ is the new coordinate of vector $\mathbf{x}$,
then 
$\mathbf{x}= \eta_1\mathbf{x}_1 + \eta_2\mathbf{x}_2+ \eta_3\mathbf{x}_3 $,
,which equals to 
\[
(\mathbf{x}_1^{T},\mathbf{x}_2^{T},\mathbf{x}_3^{T}) (\eta_1,\eta_2,\eta_3)^{T} = \mathbf{x}^{T}
\]
so $(\eta_1,\eta_2,\eta_3)^{T} 
= (\mathbf{x}_1^{T},\mathbf{x}_2^{T},\mathbf{x}_3^{T})^{-1}\mathbf{x}^{T} 
= (33,-82,154)^{T}$.\\
new coordinate of $\mathbf{x}$ : $(33,-82,154)$

\section{Problem 8 of Page 25}
let $\mathbf{x} = (\mathbf{x}_1,\mathbf{x}_2,\mathbf{x}_3,\mathbf{x}_4)$,$\mathbf{y} =(\mathbf{y}_1,\mathbf{y}_2,\mathbf{y}_3,\mathbf{y}_4)$,
then the original equation equals to
\begin{numcases}{}
\mathbf{x} (1,2,0,0)^{T} = \mathbf{y} (0,0,1,0)^{T} \\
\mathbf{x} (0,1,2,0)^{T} = \mathbf{y} (0,0,0,1)^{T}  \\
\mathbf{y} (1,2,0,0)^{T} = \mathbf{x} (0,0,1,0)^{T} \\
\mathbf{y} (0,1,2,0)^{T} = \mathbf{x} (0,0,0,1)^{T}
\end{numcases}
so

 





\end{document}
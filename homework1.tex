\documentclass[12pt,a4paper]{article}
\usepackage[utf8]{inputenc}
\usepackage{amsmath}
\usepackage{amsfonts}
\usepackage{amssymb}
\usepackage{graphicx}
\usepackage{geometry}
\usepackage{cases}
\usepackage{fontspec}
\setmainfont{Times New Roman}
\usepackage{enumerate}
\usepackage{bm}

\geometry{left=2.5cm,right=2.5cm,top=2cm,bottom=2cm}
\makeatletter
\renewcommand{\thesection}{\arabic{section}.} 
\newcommand{\mysection}[2]{
\section{Problem #1 of Page #2}	
	}
\makeatother


\title{homework1}
\author{Zuyao Chen 201728008629002 \\ zychen.uestc@gmail.com}

\date{}


\begin{document}
\maketitle
\mysection{3.(2)}{25}
yes. 
Call the set of all real symmetric matrix ``$\mathcal{S}$''.
\begin{itemize}
	\item obviously zero matrix $O \in \mathcal{S}$;
	\item $\forall A \in \mathcal{S},\forall \alpha \in \mathcal{R}$, 
			$(\alpha A)^{T} = \alpha A$, $\alpha A$ is still symmetric,
			so $\alpha A \in \mathcal{S}$;
	\item $\forall A , B \in \mathcal{S}$, $(A+B)^{T} = (A+B)$,
			$(A+B)$ is symmetric, $(A+B) \in \mathcal{S}$		
\end{itemize}
in summary,	the set of all real symmetric matrix is closed under addition and scalar multiplication, so it's a linear space.
\mysection{4}{25}
\textbf{Proof:} \\
\indent Let
$a_1 \cdot 1 + a_2 \cdot \cos^{2}t + a_3 \cdot \cos 2t = 0,a_1,a_2,a_3 \in \mathcal{R}$.
Suppose $1, \cos^{2} t, \cos 2t$ is linear independent,then 
it must has $ a_1 = a_2 = a_3 = 0$.
But as we all know,$\cos 2t = 2\cos^{2} t - 1$,
if $a_1 = 1, a_2 = -2 , a_3 = 1$, it also has $a_1 \cdot 1 + a_2 \cdot \cos^{2}t + a_3 \cdot \cos 2t = 0$,which is contrary to the hypothesis.
So $1, \cos^{2} t, \cos 2t$ is linear dependent.

\mysection{6}{25}
let $(\eta_1,\eta_2,\eta_3)$ be the new coordinates of vector $\mathbf{x}$,
then 
$\mathbf{x}= \eta_1\mathbf{x}_1 + \eta_2\mathbf{x}_2+ \eta_3\mathbf{x}_3 $,
,which equals to 
\[
(\mathbf{x}_1^{T},\mathbf{x}_2^{T},\mathbf{x}_3^{T}) (\eta_1,\eta_2,\eta_3)^{T} = \mathbf{x}^{T}
\]
so $(\eta_1,\eta_2,\eta_3)^{T} 
= (\mathbf{x}_1^{T},\mathbf{x}_2^{T},\mathbf{x}_3^{T})^{-1}\mathbf{x}^{T} 
= (33,-82,154)^{T}$.\\
new coordinates of $\mathbf{x}$ : $(33,-82,154)$

\mysection{8}{25}
\begin{enumerate}[(1)] 
	\item 
let $\mathbf{X} = (\mathbf{x}_1,\mathbf{x}_2,\mathbf{x}_3,\mathbf{x}_4)$,$\mathbf{Y} =(\mathbf{y}_1,\mathbf{y}_2,\mathbf{y}_3,\mathbf{y}_4)$,
then the original equation equals to
\begin{numcases}{}
\mathbf{X} (1,2,0,0)^{T} = \mathbf{Y} (0,0,1,0)^{T} \label{form1} \\
\mathbf{X} (0,1,2,0)^{T} = \mathbf{Y} (0,0,0,1)^{T}  \label{form2}\\
\mathbf{Y} (1,2,0,0)^{T} = \mathbf{X} (0,0,1,0)^{T} \label{form3}\\
\mathbf{Y} (0,1,2,0)^{T} = \mathbf{X} (0,0,0,1)^{T} \label{form4}
\end{numcases}
combining equation(\ref{form1}),(\ref{form2}),(\ref{form3}) and (\ref{form4}),we have \\
\begin{numcases}{ }
\mathbf{Y}(1,0,0,0)^{T} = \mathbf{X}(4,8,1,-2)^{T} \\
\mathbf{Y}(0,1,0,0)^{T} = \mathbf{X}(-2,-4,0,1)^{T} \\
\mathbf{Y}(0,0,1,0)^{T} = \mathbf{X}(1,2,0,0)^{T} \\
\mathbf{Y}(0,0,0,1)^{T} = \mathbf{X}(0,1,2,0)^{T} 
\end{numcases}
thus the transformation matrix ($\mathbf{Y}=\mathbf{XC}$):\\
\[
\mathbf{C} = \left(
\begin{array}{cccc}
4 &-2 &1 &0\\
8 &-4 &2 &1\\
1 &0 & 0 &2 \\
-2&1 & 0 &0 
\end{array}
\right)
\]	
\item let $\mathbf{z}$ be the new coordinates, \\
\[ \mathbf{z}
\left( 
\begin{array}{cccc}
4 &-2 &1 &0\\
8 &-4 &2 &1\\
1 &0 & 0 &2 \\
-2&1 & 0 &0 
\end{array}
\right)
= (2,-1,1,1) 
\]
thus,$\mathbf{z} = (2,-1,1,1) \mathbf{C}^{-1} = (-1,1,0,1) $.
\end{enumerate}
\mysection{10}{26}
call the span space ``$\mathcal{S}$'',the linear combination of 
$\mathbf{y}_1,\mathbf{y}_2,\mathbf{y}_3$ can be written as
\begin{equation} 
\begin{split}
&k_1 (\mathbf{x}_1 - 2\mathbf{x}_2 + 3\mathbf{x}_3)+
 k_2(2\mathbf{x}_1 + 3\mathbf{x}_2 + 2\mathbf{x}_3)
 + k_3(4\mathbf{x}_1 + 13\mathbf{x}_2) \\
 =& (k_1 + 2K_2 + 4k_3)\mathbf{x}_1 + 
 (-2k_1+3k_2 + 13k_3)\mathbf{x}_2 +
 (3k_1 + 2k_2)\mathbf{x}_3
\end{split}
\end{equation}
it's a linear combination of $\mathbf{x}_1,\mathbf{x}_2,\mathbf{x}_3$,
thus $\mathbf{x}_1,\mathbf{x}_2,\mathbf{x}_3$ is one basis of space $\mathcal{S}$
\mysection{11}{26} 
$\mathbf{S} = \mathbf{V}_1 \bigcap  \mathbf{V}_2 =
\{(\zeta_1,\zeta_2,\zeta_3,\zeta_4)|\zeta_1 = -\zeta_3, \zeta_2 = \zeta_4
\}
$,let $\mathbf{e}_1,\mathbf{e}_2,\mathbf{e}_3,\mathbf{e}_4$
be the standard basis of $\mathbf{R}^{4}$.
$\forall \mathbf{x} = (x_1,x_2,x_3,x_4) \in \mathbf{S}$, \\
\begin{equation}
\begin{split}
\mathbf{x} &= x_1 \mathbf{e}_1 + x_2\mathbf{e}_2 + x_3\mathbf{e}_3 + x_4\mathbf{e}_4 \\
&= x_1(\mathbf{e}_1 -\mathbf{e}_3) + x_2(\mathbf{e}_2 + \mathbf{e}_4)
\end{split} 
\end{equation} 
any element in $\mathbf{S}$ can be derived from linear combination of
$\mathbf{e}_1 - \mathbf{e}_3$ and $\mathbf{e}_2 + \mathbf{e}_4$,
so $\mathbf{e}_1 - \mathbf{e}_3$, $\mathbf{e}_2 + \mathbf{e}_4$ is one basis of $\mathbf{S}$.
\mysection{12}{26}
\begin{enumerate}[(1)]
\item \textbf{Proof}: 
\begin{itemize}
	\item obviously zero matrix $O \in \mathbf{V}$ 
	\item $\forall A,B \in \mathbf{V},\forall \alpha,\beta \in \mathbf{R},
		\alpha A + \beta B = \left(
		\begin{array}{cc}
		 \alpha a_{11} + \beta b_{11} & * \\
		 * & \alpha a_{22} + \beta b_{22}
		\end{array} 
		\right) \in \mathbf{V}
	$
\end{itemize}
$\mathbf{V}$ is closed under addition and scalar multiplication ,
thus it is a subspace of $\mathbf{R}^{2\text{x}2}$.
\item let $ 
\mathbf{e}_1 = \left(
	\begin{array}{cc}
	 1 & 0  \\
	 0 & -1  
	\end{array} 
\right),
\mathbf{e}_2 = \left(
\begin{array}{cc}
0 & 1  \\
0 & 0  
\end{array} 
\right),
\mathbf{e}_3 = \left(
\begin{array}{cc}
0 & 0  \\
1 & 0  
\end{array} 
\right)
$,
$\forall A \in \mathbf{V}$, \\
\[
A = \left( \begin{array}{cc}
a_{11} & a_{12} \\
a_{21} & -a_{11}
\end{array} \right) 
= a_{11}\mathbf{e}_1 + a_{12}\mathbf{e}_2 + 
a_{21}\mathbf{e}_3
\]
$\mathbf{e}_1,\mathbf{e}_2,\mathbf{e}_3$ is linear independent,
so the dimension of subspace $\mathbf{V}$ is 3
and $\mathbf{e}_1,\mathbf{e}_2,\mathbf{e}_3$ is one basis of $\mathbf{V}$.

\end{enumerate}
\section{Additional Problem}
\textbf{Question}: \\
if $\mathbf{W}_1,\mathbf{W}_2,\mathbf{W}_3$ are subspaces of $\mathbf{W}$, then 
$(\mathbf{W}_1 \bigcap \mathbf{W}_3) + (\mathbf{W}_2 \bigcap \mathbf{W}_3) \subset (\mathbf{W}_1 +\mathbf{W}_2) \bigcap \mathbf{W}_3$.
Can the left equal to the right under some circumstances?
 


\mysection{1.(2)}{77}
Yes. $\forall X_1 ,X_2 \in \mathbf{R}^{n\text{x}n}$,$\forall 
\alpha,\beta \in \mathbf{R}
$\\
\[T(\alpha X_1 + \beta X_2) = B(\alpha X_1 + \beta X_2)C = \alpha BX_1C + \beta BX_2C = \alpha T(X_1) + \beta T(X_2) \]
thus $T$ is linear transformation.
\mysection{6}{78}
let $A = (x_1,x_2,..,x_6)^{T}$,
\begin{equation*}
\left\lbrace 
\begin{split}
\frac{\partial x_1}{\partial t} &= ax_1-bx_2 \\
\frac{\partial x_2}{\partial t} &= bx_1+ax_2\\
\frac{\partial x_3}{\partial t} &=(1+at)x_1 - btx_2, \\
\frac{\partial x_4}{\partial t}&=btx_1 + (1+at)x_2,\\
\frac{\partial x_5}{\partial t}&=(t+\frac{1}{2}at^{2})x_1
- \frac{1}{2}bt^{2}x_2,\\
\frac{\partial x_6}{\partial t}&= \frac{1}{2}bt^{2}x_1 + (t+\frac{1}{2}at^{2})x_2
\end{split}
\right .
\end{equation*}
thus 
\begin{equation*}
\begin{split}
\nabla_t(A) &= (\frac{\partial x_1}{\partial t},\frac{\partial x_2}{\partial t},...,\frac{\partial x_6}{\partial t})^{T} \\
&= \left(
\begin{array}{cccccc}
a & -b & 0 & 0 & 0 &0 \\
b & a & 0 &0 &0&0\\
1+at&-bt&0&0&0&0\\
bt&1+at&0&0&0&0 \\
t+\frac{1}{2}at^{2}&-\frac{1}{2}bt^{2}&0&0&0&0 \\
\frac{1}{2}bt^{2}&t+\frac{1}{2}at^{2}&0&0&0&0
\end{array}
\right)A \\
&= D A
\end{split}
\end{equation*}
$\forall B = (b_1, b_2 , ...
,b_6)^{T} \in \mathbf{V}$,
\[
\nabla_t(B) = D B
\]


\end{document}
\documentclass[12pt,a4paper]{article}
\usepackage[utf8]{inputenc}
\usepackage{amsmath}
\usepackage{amsfonts}
\usepackage{amssymb}
\usepackage{graphicx}
\usepackage{geometry}
\usepackage{cases}
\usepackage{fontspec}
\setmainfont{Times New Roman}
\usepackage{enumerate}
\usepackage{bm}
\usepackage{tikz}
\usetikzlibrary{backgrounds}

\usepackage{xcolor}
\usepackage{float}
\usepackage{subfigure}

\geometry{left=1in,right=1in,top=1in,bottom=1in}
\makeatletter
\renewcommand{\thesection}{\arabic{section}.} 
%
\newcommand{\mysection}[2]{
\section{Problem #1 of Page #2}	
	}

%\renewcommand{\thesubsection}{(\arabic{subsection})} 

\makeatother


\title{homework3}
\author{Zuyao Chen 201728008629002 \\ zychen.uestc@gmail.com}

\date{}


\begin{document}
\maketitle
\mysection{1}{195}
$\Delta_1 = 5, \Delta_2 =1, \Delta_3 = 1,\Delta_4 = -7$, 
\[
	L_1 = \left( \begin{array}{cccc}
		1 & & & \\
		\frac{2}{5} & 1 & &  \\
		\frac{-4}{5} & &1 &  \\
		0 & & & 1
	\end{array} \right),
	L_1^{-1} = \left( \begin{array}{cccc}
	1 & & & \\
	\frac{-2}{5} & 1 & &  \\
	\frac{4}{5} & &1 &  \\
	0 & & & 1
	\end{array} \right)
\]
then
\[	
	A^{(1)} = L_1^{-1}A^{(0)} = \left( \begin{array}{cccc}
	5 & 2 & -4 & 0\\
	0 & 1/5 & -2/5 & 1 \\
	0 & -2/5 & 9/5 & 0 \\
	0 & 1 & 0 & 2 
	\end{array} \right)
\]
\[
	L_2 = \left( \begin{array}{cccc}
	1 & & & \\
	 & 1 & &  \\
	 & -2&1 &  \\
	 & 5& & 1
	\end{array} \right),
	L_2^{-1} = \left( \begin{array}{cccc}
	1 & & & \\
	& 1 & &  \\
	& 2&1 &  \\
	& -5& & 1
	\end{array} \right)	
\]
\[
	A^{(2)} = L_2^{-1}A^{(1)} = \left( \begin{array}{cccc}
	5 & 2 & -4 & 0\\
	0 & 1/5 & -2/5 & 1 \\
	0 & 0 & 1 & 2 \\
	0 & 0 & 2 & -3 
	\end{array} \right)
\]
\[
	L_3 = \left( \begin{array}{cccc}
	1 & & & \\
	& 1 & &  \\
	& &1 &  \\
	& &2& 1
	\end{array} \right),
	L_3^{-1} = \left( \begin{array}{cccc}
	1 & & & \\
	&  & &  \\
	& &1 &  \\
	& &-2 & 1
	\end{array} \right)	
\]
\[ \begin{split} 
	A^{(3)} &= L_3^{-1}A^{(2)} =  \left( \begin{array}{cccc}
	5 & 2 & -4 & 0\\
	0 & 1/5 & -2/5 & 1 \\
	0 & 0 & 1 & 2 \\
	0 & 0 & 0 & -7 
	\end{array} \right) \\
	&= \text{diag}(5,1/5,1,-7) \left( \begin{array}{cccc}
	1 & 2/5 & -4/5 & 0\\
	0 & 1/5 & -2 & 5 \\
	0 & 0 & 1 & 2 \\
	0 & 0 & 0 & 1 
	\end{array} \right) = DU
	\end{split} 
\]
thus 
\[
	L = L_1L_2L_3 =  \left( \begin{array}{cccc}
	1 &   &   & \\
	2/5 & 1 &  & \\
	-4/5 & -2 & 1 &  \\
	0 & 5 & 2 & 1 
	\end{array} \right)
\]
Doolittle decomposition:
\[
	A = L(DU) = L\hat{U}
\]
here 
\[
	\hat{U} = DU = \left( \begin{array}{cccc}
	       5        &     2      &      -4     &        0  \\    
	       0       &     1/5     &     -2/5    &        1  \\    
	       0       &      0      &       1     &        2  \\    
	       0       &      0      &      0      &      -7  
	    \end{array}\right)
\]
\mysection{3}{195}
\textbf{Proof:} \\
since
\[
	L_1 = \left(\begin{array}{cccc} 
	 1 & & & \\
	 c_{21} & 1& & \\
	 \vdots & & \ddots & \\
	 c_{n1} & & &1
	\end{array} \right) ,
	L_1^{-1} = \left(\begin{array}{cccc} 
		1 & & & \\
		-c_{21} & 1& & \\
		\vdots & & \ddots & \\
		-c_{n1} & & &1
		\end{array} \right) 
\]
here $c_{i1} = a_{i1}/a_{11} \ ( i=2,3,...,n)$,
\[
	A^{(1)} = L_1^{-1}A^{(0)} = \left(\begin{array}{cccc} 
	a_{11} & a_{12} & \cdots & a_{1n} \\
	0 & a_{22} & \cdots & a_{2n} \\
	0 & \vdots & & \vdots \\
	0 & a_{n2} & \cdots & a_{nn}
	\end{array}\right)
\]
thus 
\[
	B = \left( \begin{array}{cccc}
	 a_{22} & a_{23} & \cdots & a_{2n} \\
	 a_{32} & a_{33} & \cdots & a_{3n} \\
	 \vdots & & \vdots \\
	 a_{n2} & a_{n3} & \cdots & a_{nn}
	\end{array} \right) 
\]
since $A$ is a real symmetric positive-definite matrix,
$B$ is a real symmetric positive-definite matrix and its diagonal elements 
are unchanging.

\mysection{4}{195}
$\Delta_1 = 5, \Delta_2 = 1, \Delta_3 =  1$,
\[
	L_1 = \left(\begin{array}{ccc}
	 1 & & \\
	 2/5 & 1 & \\
	 -4/5 & & 1 	
	\end{array} \right),
	A^{(1)} = L_1^{-1} A^{(0)} = \left(\begin{array}{ccc}
	5 & 2 & -4 \\
	0 & 1/5 & -2/5\\
	0 & -2/5& 9/5 	
	\end{array} \right)
\]
\[
	L_2 = \left(\begin{array}{ccc}
	1 & & \\
     & 1 & \\
	  &-2 & 1 	
	\end{array} \right),
	A^{(2)} = L_2^{-1} A^{(1)} = \left(\begin{array}{ccc}
	5 & 2 & -4 \\
	0 & 1/5 & -2/5\\
	0 & 0 & 1 	
	\end{array} \right)	
	= \text{diag}(5,1/5,1) \left(\begin{array}{ccc}
	1 & 2/5 & -4/5 \\
	0 & 1 & -2\\
	0 & 0 & 1 	
	\end{array} \right)	
\]
thus 
\[ \begin{split} 
	G &= L\tilde{D} = L_1L_2 \text{diag}(\sqrt{5}, \sqrt{1/5},1) \\
	&= \left(\begin{array}{ccc}
	\sqrt{5} &  &  \\
	2/\sqrt{5} & 1/\sqrt{5} & \\
	-4/\sqrt{5} & -2/\sqrt{5} & 1 	
	\end{array} \right)	
	\end{split} 
\]
\[
	A = GG^T
\]
\mysection{2}{219}
$T_{12}(c,s): c= 2/\sqrt(13), s = 3/\sqrt(13)$,
\[
	T_{12} = \left(\begin{array}{cccc}
		2/\sqrt{13} & 3/\sqrt{13} &  & \\
		-3/\sqrt{13} & 2/\sqrt{13} & & \\
		 &  & 1 & \\
		 & & & 1 
	\end{array} 	\right),
	T_{12}\bm{x} =  (\sqrt{13}, 0,0,5)^T
\]
$T_14(c,s) : c= \sqrt{13}/\sqrt{38}, s = 5/\sqrt{38}$,
\[
	T_{14} = \left(\begin{array}{cccc}
	\sqrt{13}/\sqrt{38} &   &  &5/\sqrt{38} \\
	  & 1 & & \\
	&  & 1 & \\
	-5/\sqrt{38}& & & \sqrt{13}/\sqrt{38}
	\end{array} 	\right),
	T_{12}\bm{x} =  (\sqrt{38}, 0,0,0)^T	
\]
thus
\[
	T = T_{14}T_{12} = \left(\begin{array}{cccc}  
	2/\sqrt{38} & 3/\sqrt{38} & 0 & 5/\sqrt{38} \\
	-3/\sqrt{13} & 2/\sqrt{13} & 0 & 0 \\
	0 & 0 &1 &0 \\
	-10/\sqrt{13*38} & -15/\sqrt{13*38} & 0 & \sqrt{13/38} 
	\end{array} \right) ,
	T\bm{x} = \sqrt{38}\bm{e}_1
\]
\mysection{4}{219}
\[ \begin{split} 
		(\bm{Hx})^T &= \bm{x}^T - a(\bm{x},\bm{w})\bm{w}^T \\
		\bm{Hx}(\bm{Hx})^T &= (\bm{x} - a(\bm{x},\bm{w})\bm{w})(\bm{x}^T - a(\bm{x},\bm{w})\bm{w}^T) \\
		&= \bm{xx}^T - 2a(\bm{x},\bm{w})\bm{wx}^T + a^2(\bm{x},\bm{w})^2 \bm{ww}^T \\
		&= \bm{xx}^T - 2a(\bm{wx}^T)^2 + a^2(\bm{wx}^T)^2
	\end{split}
\]
let $\bm{H}\bm{H}^T = \bm{H}^T\bm{H} = I$,
\[
	- 2a(\bm{wx}^T)^2 + a^2(\bm{wx}^T)^2 = 0	
\]
hence $a = 0 , 2$.

\mysection{7}{219}
\[
	b_1 = (2,0,2)^T, T_{13} = \left( \begin{array}{ccc}
	1/\sqrt{2} & 0 & 1/\sqrt{2} \\
	0 & 1  & 0\\
	-1/\sqrt{2} & 0 & 1/\sqrt{2}
	\end{array} \right) 
\]
then
\[
	T_{13}A^{(0)} = \left( \begin{array}{ccc}
	4/\sqrt{2} & 3/\sqrt{2} & 3/\sqrt{2} \\
	0 & 2 & 2 \\
	0 & -1/\sqrt{2} & 1/\sqrt{2} 
	\end{array}\right)
\]
\[
	b_2 = (2,-1/\sqrt{2})^T, T_{12} = \left(\begin{array}{cc}
	 2\sqrt{2}/3 & -1/3 \\
	 1/3 & 2\sqrt{2}/3
	\end{array} \right) 
\]
\[
	T_{12}A^{(1)} = \left( \begin{array}{cc}
	 3\sqrt{2}/2 & 7\sqrt{2}/6 \\
	 0 & 4/3
	\end{array} \right) 
\]
\[
	T= \left( \begin{array}{cc}
	1 & \\
	 & T_{12}
	\end{array} \right)T_{13}
	= \left(\begin{array}{ccc} 
	 1/\sqrt{2} & 0 & 1/\sqrt{2} \\
	 1/(3\sqrt{2}) & 2\sqrt{2}/3 & -1/(3\sqrt{2}) \\
	 -2/3 & 1/3 & 2/3	
	\end{array} \right) 
\]
\[
	Q = T^T,
	R = \left(\begin{array}{ccc} 
		4/\sqrt{2} & 3/\sqrt{2} & 3/\sqrt{2} \\
			& 3\sqrt{2}/2 & 7\sqrt{2}/6 \\
			& 0 & 4/3	
	\end{array} \right) 
\]

\mysection{8}{219}
\[
	\bm{b}_1 = (0,1,0)^T, \bm{b}_1 - |\bm{b}_1|\bm{e}_1 = (-1,1,0)^T,
	\bm{u} = \frac{1}{\sqrt{2}}(-1,1,0)^T
\]
then
\[
	H_1 = \bm{I} - 2\bm{uu}^T = \left(\begin{array}{ccc}
		 0& 1 &0 \\
		 1& 0 &0 \\
		 0& 0 & 1
	\end{array} \right),
	H_1 A^{(0)} = \left( \begin{array}{ccc}
		1 & 1 & 1 \\
		0 & 4 & 1\\
		0 & 3 & 2
	\end{array} \right) 
\]
\[
	\bm{b}_2 = (4,3)^T,  \bm{b}_2 - |\bm{b}_2|\bm{e}_1 = (-1,3)^T,
	\bm{u} = \frac{1}{\sqrt{10}}(-1,3  )^T
\]
\[
	\bm{H}_2 = \bm{I} - 2\bm{uu}^T= \left(\begin{array}{cc}
	4/5& 3/5 \\
	3/5& -4/5 \\
	\end{array} \right), 
	H_2A^{(1)} = \left( \begin{array}{cc}
		5 & 2 \\
		0 & -1
	\end{array} \right) 
\]
Hence
\[
	H = \left( \begin{array}{cc}
		1 & \\
		 & H_2 
	\end{array}\right) 	H_1 = \left(\begin{array}{ccc}
		0 & 1 & 0 \\
		4/5 & 0 & 3/5 \\
		3/5 & 0 &-4/5 \\
	\end{array}\right) ,
	Q = H^T, 
	R = \left( \begin{array}{ccc}
		1 & 1 & 1 \\
		& 5 & 2\\
		& & -1
	\end{array} \right) 
\]
\mysection{1}{225}
\begin{enumerate}[(1)]
	\item  \[
		A \rightarrow B = \left( \begin{array}{cccc}
		1 & 0 & 2 & 1\\
		0 & 1 & 1/2 & -1/2 \\
		0 & 0 & 0 & 0	
	\end{array} 		\right) 
	\]
  take $a_1,a_2$, 	
  \[
	  F = (a_1, a_2) = \left( \begin{array}{cc}
	  1 & 2\\
	  0 & 2\\
	  1 & 0 \\
	  \end{array}\right) ,
	  G = \left( \begin{array}{cccc}
		  1 & 0 & 2 & 1\\
		  0 & 1 & 1/2 & -1/2
	  \end{array} \right) 
  \]
 \item 
 \[
 A \rightarrow B = \left( \begin{array}{cccc}
 1 & 0 & 0 & 0\\
 0 & 1 & -1 & -1 \\
 0 & 0 & 0 & 0 \\
 0 & 0 & 0 & 0 
 \end{array} 		\right) 
 \]
 take $a_1,a_2$, 	
 \[
 F = (a_1, a_2) = \left( \begin{array}{cc}
 1 & -1\\
 -1 & 1\\
 -1 & -1 \\
 1 & 1
 \end{array}\right) ,
 G = \left( \begin{array}{cccc}
 1 & 0 & 0 & 0\\
 0 & 1 & -1 & -1 \\
 \end{array} \right) 
 \]
	
\end{enumerate}
\mysection{2}{225}
\textbf{Proof:} \\
since $\text{rank}(\bm{B}) = r$ ,we have
\[
	\bm{B} = \bm{QR}
\]
here $\bm{Q} \in \mathbb{R}_r^{m\times r}, \bm{R} \in \mathbb{R}_r^{r\times r}$ and
$\bm{Q}^T\bm{Q} = \bm{I}$, $\bm{R}$ is a non-singular triangular matrix.Hence
\[
	\bm{B}^T\bm{B} = (\bm{QR})^T\bm{QR} = \bm{R}^T\bm{Q}^T\bm{QR} = \bm{R}^T\bm{R}
\]
$\bm{B}^T\bm{B}$ is non-singular.
\mysection{3}{225}
\textbf{Proof:}
\begin{itemize}
	\item if $\text{rank}(\bm{A}) = m $, then we have $\bm{A} = \bm{QR}, \bm{Q} \in \mathbf{C}_m^{n\times m} , \bm{Q}^H\bm{Q}=\bm{I},
	\bm{R} \in \mathbf{C}_m^{m\times m}$,
	 let $\bm{B} = \bm{R}^{-1}\bm{Q}^H \in \mathbf{C}^{m\times n}$, then $\bm{BA} = \bm{R}^{-1}\bm{Q}^H\bm{QR} = \bm{I}$;
	\item if $\bm{BA} = \bm{I}$, let $
	 \text{rank}(\bm{B}) = r \leq n$,
	then $ \bm{B} = \bm{Q}\bm{R}, 
	 \bm{Q} \in \mathbf{C}_{r}^{m\times {r}}, \bm{R}\in \mathbf{C}_{r}^{{r}\times n} 
	 $,
	 \[ \begin{split} 
		 \bm{BA} &= \bm{Q}\bm{R}\bm{A}= \bm{I} \\
		 \bm{R}\bm{A} &= \bm{Q}^H
		\end{split} 
	 \]
	  $\text{rank}(\bm{RA}) = r $ and $\text{rank}(\bm{R}) = r$, 
	 thus $\text{rank}(\bm{A}) = m$.
\end{itemize}
	
\mysection{4}{225}
\textbf{Proof:}\\
$\bm{F} = \bm{Q}_1\bm{R}_1, \bm{Q}_1\in\mathbf{C}_r^{m\times r}, \bm{R}_1\in\mathbf{C}_r^{r\times r}$,
\[
	\bm{FG} = \bm{Q}_1\bm{R}_1\bm{G} = \bm{Q}_1 \bm{R}
\]
$\bm{R} = \bm{R}_1\bm{G}\in \mathbf{C}_r^{r\times n}$,thus
\[
	\text{rank}(\bm{FG}) = r
\]

\mysection{1}{233}



\mysection{3}{233}

\mysection{4}{233}

\mysection{5}{233} 

\end{document}